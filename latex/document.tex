% durch Austauschen dieser Zeilen kann die Sprache des Templates geändert werden
\PassOptionsToPackage{main=ngerman}{babel}
%\PassOptionsToPackage{main=english}{babel}

% durch Austauschen dieser Zeilen kann zwischen Abschlussarbeit und Seminararbeit gewechselt werden
\documentclass[thesis]{mi-document}
%\documentclass[seminar]{mi-document}

\bachelor % im Falle einer Masterarbeit \master
%\master

% Variablen, die für das Deckblatt und Metadaten verwendet werden
\title{[Titel der Bachelor-/ Masterarbeit]}
\author{[Autor*in der Arbeit]}
\semester{[WS / SS und Jahreszahl]}
\course{[Art des Seminars und Seminartitel (z.B. Praktikum Multimedia Engineering)]}
\module{[z.B. MEI-M 04 (B.A.)]}
\dozent{[Seminarleiter]}
\studid{[Matrikelnummer]}
\studSemester{[Semesterzahl und Studiengänge (z.B. 3. Semester B.A. Medieninformatik / Informationswissenschaft)]}
\phone{0941/133742666} % Optional
\studSubject{Medieninformatik}
\firstReviewer{Prof. Dr. Maike Musterprof}
\secondReviewer{Prof. Dr. Max Musterprof}
\advisor{Momo Mustermensch}
\address{Domplatz 1, 93047 Regensburg}{} % Optional
\mail{[Emailadresse (z.B.: max.mustermann@stud.uni-regensburg.de)]}
\studMail{[Emailadresse (z.B.: max.mustermann@stud.uni-regensburg.de)]}
\dateHandedIn{[Abgabetermin der Arbeit]}
\keywords{Enter;key;words;here}

\bibliographystyle{apacite}

% Falls Sie die Abkürzung zum Einbinden von Grafiken benutzen möchten. Erläuterung fnden Sie im Abschnitt zu Abbildungen.
\input{config}

\begin{document}

% auskommentieren, damit Sachen nach Kapitel nummeriert werden (z.B. Abbildung 3.2)
\counterwithout{footnote}{chapter}
\counterwithout{figure}{chapter}
\counterwithout{table}{chapter}
\counterwithout{lstlisting}{chapter}

% Die Nummerierung beginnt mit der Titelseite (= Seite 1), soll aber erst ab der ersten Inhaltsseite (Einleitung) angezeigt werden.
\pagestyle{empty}

% Deckblatt des Templates und Hinweise
% diese Zeile für die Verwendung des Templates entfernen!
\input{hinweise}

% Das Deckblatt erstellen
\maketitle

\tableofcontents % Optional
\listoffigures % Optional
\listoftables % Optional
\lstlistoflistings % Optional


\clearpage
\doublespacing

\input{abstract}

\clearpage
\pagestyle{headings} % Seitennummern und Kapitelbezeichnungen anzeigen

% hier beginnt der eigentliche Inhalt der Arbeit

\include{aufgabenstellung}
\include{einleitung}
\include{ziele}
\include{stand_der_technik}
\include{gestaltungsrichtlinien}
\include{empfehlungen}
\include{zusammenfassung}

% Kapitelbezeichnung in der rechten oberen Ecke entfernen
\clearpage
\pagestyle{plain}

% Literaturverzeichnis anzeigen
% kleinerer Zeilenabstand, damit es nicht so gestreckt aussieht
\onehalfspacing
\bibliography{literature}
\doublespacing

% Anhang
\appendix
\include{anhang}

% Tipps zur Verwendung von LaTeX
\include{latex}

\begin{singlespace}
\KOMAoptions{parskip=full}
% Erklärung zur Urherberschaft (urheberschaft.tex) anhängen
\addchap{Erklärung zur Urheberschaft}

Ich habe die Arbeit selbständig verfasst, keine anderen als die angegebenen Quellen und Hilfsmittel benutzt, sowie alle Zitate und Übernahmen von fremden Aussagen kenntlich gemacht. 

Die Arbeit wurde bisher keiner anderen Prüfungsbehörde vorgelegt. 

Die vorgelegten Druckexemplare und die vorgelegte digitale Version sind identisch.

\ifdefstring
{\getWorkType}{Masterarbeit}
{Von den zu § 27 Abs. 5 der Prüfungsordnung vorgesehenen Rechtsfolgen habe ich Kenntnis.}
{}

\ifdefstring
{\getWorkType}{Master's Thesis}
{Von den zu § 27 Abs. 5 der Prüfungsordnung vorgesehenen Rechtsfolgen habe ich Kenntnis.}
{}

\signature


% Erklärung zur Lizenzierung der Arbeit (lizenzierung.tex) anhängen
\include{lizenzierung}

% Stichwortverzeichnis anzeigen. Weiß nicht, warum das nicht nach dem Inhaltsverzeichnis kommt.
\printindex
\end{singlespace}

% Inhalt des Datenträgers. Gehört meiner Meinung nach zum Anhang, aber was weiß ich schon. AS
\newpage
\input{datenträger}

\end{document}
